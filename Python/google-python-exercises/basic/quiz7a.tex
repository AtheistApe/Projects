\documentclass[10pt,fleqn]{article}

%       amslatex provides nice math extensions for typesetting mathematics
\usepackage{amsmath}
\usepackage{tmmaths}

%       pstricks provides powerful environments for incorporating postscript into a 
%       TeX/LaTeX document. You must have a postscript printer and a package like
%       dvips to convert the DVI file to a PS file.
%\usepackage{pst-all}
%\usepackage{pstricks,pst-plot}
%\usepackage{pst-coil,pst-node}
\usepackage{tkz-tab,tkz-euclide,tkz-fct}
\usetkzobj{all}   
\usepackage{calc}

%  This package provides native tex support for numbered grids. The syntax is:
%  \graphpaper[spc](x_lowleft,y_lowleft)(x_upperright,y_upperright)

%\usepackage{graphpap}
%\usepackage{float}

%  The package below must be initialized with "\initfloatingfigs" immediately after the
%  "\begin{document} command.
%\usepackage{floatfig}

\usepackage{graphicx}
\graphicspath{{i:/mytex/graphics}}
\DeclareGraphicsExtensions{.ps,.eps}

%       tst is a package for the creation of exams, quizzes and tests. the include
%       file mathstuf (see below) provides many abbreviations for these environments.
%\usepackage{tst}

%       epsfig is a package which provides for the inclusion of Encapsulated PostScript
%       files in a document.
%\usepackage{epsfig}
%\usepackage{epic,eepic}
\include{mathstuf}
\usepackage[total={7.75in,10in},top=0.25in,left=0.5in,includehead]{geometry}
\usepackage{fancyhdr}
\pagestyle{fancy}
\lhead{Math 120}
\rhead{\large Name\makebox[2in]{\hrulefill}}
\chead{\LARGE Quiz 7}
\lfoot{\today}
\cfoot{}
%\rfoot{\thepage}
\renewcommand{\headrulewidth}{0.4pt}
\renewcommand{\footrulewidth}{0.4pt}
\setlength{\parindent}{0pt}
\setlength{\parskip}{2ex}

\newcounter{tf}[enumi]
\newenvironment{tf}[0]{\begin{list}%
{\alph{tf}. \makebox[5em]{True\hfill False}}%
{\usecounter{tf}\setlength{\labelwidth}{7em}%
\setlength{\leftmargin}{3.5cm}%
\setlength{\labelsep}{1cm}}}%
{\end{list}}

\newcommand{\ds}{\displaystyle}
\usepackage{tabularx}
\begin{document}
Carefully read the instructions to each problem in the quiz and
respond directly to the questions asked. Work the problems in as neat
and organized a manner as possible. Clearly show your work and circle
or otherwise indicate your answers. Good success to you.
\begin{enumerate}
\begin{minipage}[t]{0.5\textwidth}
  \vspace{0pt}
\item Use the graph at the right to do the following
  problems. The plotted curve is the graph of a function $y = f(x)$.
  \begin{enumerate}
    \item Does this function have an inverse function $f^{-1}$? How
      can you tell? Be specific!\\[0.5in]
    \item What is the approximate value of $f(2)$ to the nearest tenth?\\[0.25in]
    \item For what value(s) of $x$ is $f(x) = 1$?
  \end{enumerate}
\end{minipage}
\hfill
\begin{minipage}[t]{0.4\textwidth}
  \vspace{-2ex}
\begin{tikzpicture}[xscale=0.75]
  \tkzInit[xmin=-5,xmax=5,ymin=-2,ymax=5]
  \tkzGrid
  \tkzAxeXY
  \tkzFct[color=black,line width=2pt]{0.25*(x-1)*(x+3)*x + 1}
  \tkzText[right](2,3.5){$y = f(x)$}
\end{tikzpicture}
\end{minipage}\\[0.25in]
\begin{minipage}[t]{0.5\textwidth}
  \vspace{0pt}
\item Use the graph at the right to do the following
  problems. The curve already plotted has equation $y = b^x$,
  for some value of $b$.
  \begin{enumerate}
    \item Is the base $b$ a number bigger than one or a number
      between zero and one? How can you tell?\\[0.5in]
    \item What is the approximate value of $b$ to the nearest
      tenth.\\[0.25in]
    \item Is this an increasing or decreasing exponential
      function?
  \end{enumerate}
\end{minipage}
\hfill
\begin{minipage}[t]{0.4\textwidth}
  \vspace{-2ex}
\begin{tikzpicture}[xscale=0.75]
  \tkzInit[xmin=-5,xmax=5,ymin=-2,ymax=5]
  \tkzGrid
  \tkzAxeXY
  \tkzFct[color=black,line width=2pt]{1.5**x}
  \tkzText[left](3,3.5){$f(x) = b^{x}$}
\end{tikzpicture}
\end{minipage}
\item If $f$ is an invertible function with inverse $f^{-1}$ and the point $(-2,7)$
  is on the graph of $f$, then:
  \begin{enumerate}
    \item name one point which must be on the graph of $f^{-1}$.
    \item What is the value of $(f^{-1} \circ f)(-2)$?
  \end{enumerate}
\item The function $g(x) = \frac{1}{3} x - 1 = y$ is a one-to-one function. Find a formula
  for $g^{-1}(x)$.
\end{enumerate}

\end{document}
